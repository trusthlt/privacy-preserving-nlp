% !TeX document-id = {e2312f8c-b2d9-4f5f-a6e9-54d5ee345192}
% !TeX program = lualatex
% !BIB program = biber
% Lualatex is important to render TTF fonts; with pdflatex it's just the regular one
% ratio 16:9 -- https://tex.stackexchange.com/questions/14336/

% compile two versions, inspired by https://tex.stackexchange.com/a/1501
% use the script "compile-pdf.sh"
\newif\ifhandout
% if flags.tex does not exist, create an empty file to be able to compile in TeXstudio
\input{flags}

\ifhandout
\documentclass[12pt,aspectratio=169,handout]{beamer}
\else
\documentclass[12pt,aspectratio=169]{beamer}
\fi

\input{header-include.tex}


\title{Privacy-Preserving Natural Language Processing}
\subtitle{Lecture 10 : Real-World DP Application and the Epsilon Registry}
\date{July 17, 2025}
\author{Dr.\ Erion {\c C}ano}
\institute{
\texttt{www.trusthlt.org} \\
Chair of Trustworthy Human Language Technologies (TrustHLT) \\
Ruhr University Bochum \& Research Center Trustworthy Data Science and Security}

\begin{document}


\maketitle


\section{Tech Giants Applying DP}

%-----------------------------------------------------------------------

\begin{frame}{Apple User Data from macOS and iOS}

Tons of user data being collected by Apple:

\begin{itemize} \setlength\itemsep{2mm}
%
\item Typing information for QuickType suggestions
%
\item Emojis for emoji suggestions
%
\item Web navigation data from Safari
%
\end{itemize}

Apple uses local DP with different $\varepsilon$ values to protect user privacy in those data. \vskip1mm

Full details are reported {\it \href{https://docs-assets.developer.apple.com/ml-research/papers/learning-with-privacy-at-scale.pdf}{here}}.

\end{frame}

%-----------------------------------------------------------------------

\begin{frame}{Meta and Facebook User Interactions}

Dataset of user interactions with web pages shared on FB is publicly available. \vskip1mm

Examples: \vskip1mm 

\begin{itemize} \setlength\itemsep{2mm}
%
\item John shared URL foo
%
\item Mary views post with URL bar
%
\end{itemize}

Meta uses $(\varepsilon,\delta)$ DP to protect each interaction.

\begin{tikzpicture}[overlay, remember picture]
\node at (current page.north east)[ref] {
\fullcite{DVN/TDOAPG_2020} \par};
\end{tikzpicture}

\end{frame}

%-----------------------------------------------------------------------

\begin{frame}{Microsoft and Windows Telemetry}

User data collections starting (massively) with Windows 10. \vskip1mm

Examples: \vskip1mm 

\begin{itemize} \setlength\itemsep{2mm}
%
\item Application crash reports. 
%
\item User time spent on certain apps.
%
\item Typing, location, connected devices, etc. 
%
\end{itemize}

Microsoft uses local DP with $\varepsilon = 1.67$ to protect it before using / sharing / selling.

\begin{tikzpicture}[overlay, remember picture]
\node at (current page.north east)[ref] {
\fullcite{ding2017collecting} \par};
\end{tikzpicture}

\end{frame}

%-----------------------------------------------------------------------

\begin{frame}{Google Shopping and Trends}

Product page view counts for prioritizing page crawing. 

\begin{itemize} \setlength\itemsep{2mm}
%
\item Collected from users on daily basis. 
%
\item Streammed using DP-SQLP
%
\item Protected using $(\varepsilon, \delta)$ DP with $\varepsilon = 1$ and $\delta = 10^{-9}$
%
\end{itemize}

Search trends used to show related queries. 
 
\begin{itemize} \setlength\itemsep{2mm}
%
\item Streammed using DP-SQLP
%
\item Protected using $(\varepsilon, \delta)$ DP with $\varepsilon = 2$ and $\delta = 10^{-10}$
%
\end{itemize}

\end{frame}

%-----------------------------------------------------------------------

\begin{frame}{Linkedin Audience Engagement}

Interactive query system that allows marketers to get information about LinkedIn users engaging with their content. 

\begin{itemize} \setlength\itemsep{2mm}
%
\item Each analyst sends queries to the system 
%
\item Each query returns $(\varepsilon, \delta)$ DP with $\varepsilon = 0.15$ and $\delta = 10^{-10}$
%
\item Monthly limit (budget) for queries is $\varepsilon = 34.9$ and $\delta = 7 \cdot 10^{-9}$
%
\item Extra measures to prevent averaging attacks 
\end{itemize}

\begin{tikzpicture}[overlay, remember picture]
\node at (current page.north east)[ref] {
\fullcite{DBLP:journals/corr/abs-2002-05839} \par};
\end{tikzpicture}

\end{frame}

%-----------------------------------------------------------------------

\begin{frame}{Labor Market Insights}

Data from LinkedIn to measure trends in people changing their occupation. \vskip2mm 

\begin{itemize} \setlength\itemsep{2mm}
%
\item Companies who are hiring, using with $\varepsilon = 14.4$ and $\delta = 1.2 \cdot 10^{-9}$
%
\item Available jobs, protecting hiring event with $\varepsilon = 14.4$ and $\delta = 1.2 \cdot 10^{-9}$
%
\item Required skills, protecting each user's skills information with $\varepsilon = 0.3$ and $\delta = 3 \cdot 10^{-10}$
%
\item Extra measures to prevent averaging attacks 
\end{itemize}

\begin{tikzpicture}[overlay, remember picture]
\node at (current page.north east)[ref] {
\fullcite{rogers2020membersapproachenablinglinkedins} \par};
\end{tikzpicture}

\end{frame}

%-----------------------------------------------------------------------

\section{Other Organizations Applying DP}

%-----------------------------------------------------------------------

\begin{frame}{USA Census Bureau}

\textbf{County Business Patterns} \vskip1mm

\begin{itemize} \setlength\itemsep{2mm}
%
\item Business establishments in the USA
%
\item Different types of data (including finances) 
%
\item Protected with various $(\varepsilon, \delta)$ parameters based on:
%
	\begin{itemize} \setlength\itemsep{2mm}
	%
	\item Financial assets
	%
	\item Annual payroll 
	%
	\item Number of employs 
	%
	\end{itemize}
%
\end{itemize}

Full details are reported {\it \href{https://www.census.gov/topics/business-economy/disclosure/data/tables/cbp-privacy-demonstration-tables.html}{here}}.

\end{frame}

%-----------------------------------------------------------------------

\begin{frame}{USA Census Bureau}

\textbf{The 2020 USA Census} \vskip1mm

\begin{itemize}
%
\item Collecting demographic information about USA population \vskip1mm
%
\item Datasets protected with various $(\varepsilon, \delta)$ parameters \vskip1mm
%
	\begin{itemize} \setlength\itemsep{2mm}
	%
	\item {\it \href{https://www.census.gov/programs-surveys/decennial-census/about/rdo/summary-files.html}{Redistricting}} data protected with $\varepsilon = 13.64$
	%
	\item {\it \href{https://www.census.gov/data/tables/2023/dec/2020-census-dhc.html}{Demographic Housing}} protected with $\varepsilon = 19.46$
	%
	\item Racial and ethnical categories protected with $\varepsilon = 45.68$
	%
	\item Suplemental household statistics protected with $\varepsilon = 12.74$
	%
	\end{itemize}
%
\end{itemize}

\end{frame}

%-----------------------------------------------------------------------

\begin{frame}{USA Census Bureau}

\textbf{Post-Secondary Employment Outcomes} \vskip2mm

\begin{itemize} \setlength\itemsep{2mm}
%
\item Earnings and employment of college graduates
%
\item Categorized by degree level, degree major, and post-secondary institution
%
\item Matching university granscript data with national database of jobs
%
\item Each person protected with $\varepsilon = 1.5$
%
\end{itemize}

Full details are reported {\it \href{https://lehd.ces.census.gov/doc/PSEOTechnicalDocumentation.pdf}{here}}.

\end{frame}

%-----------------------------------------------------------------------

\begin{frame}{Wikimedia Foundation}

\textbf{Page view statistics} \vskip2mm

\begin{itemize} \setlength\itemsep{2mm}
%
\item Distinct users visiting Wiki pages each day from each country
%
\item July 1, 2015 to Feb. 8, 2017 protected with $\varepsilon = 1$ on 300 pages view per day
%
\item Feb. 9, 2017 to Feb. 5, 2023 protected with $\varepsilon = 1$ on 30 pages view per day
%
\item Feb. 6, 2023 onwards protected with $\varepsilon = 0.72$ and $\delta = 10^{-5}$
%
\end{itemize}

\begin{tikzpicture}[overlay, remember picture]
\node at (current page.north east)[ref] {
\fullcite{adeleye2023publishingwikipediausagedata} \par};
\end{tikzpicture}

\end{frame}

%-----------------------------------------------------------------------

\begin{frame}{Wikimedia Foundation}

\textbf{Editor statistics} \vskip2mm

\begin{itemize} \setlength\itemsep{2mm}
%
\item Statistics about editor activity by project and country \vskip2mm
%
\item Monthly published data protected with $\varepsilon = 2$ on editor-project-country-month
%
\item Weekly published data protected with $\varepsilon = 2$ on editor-project-country-week
%
\item One-off release for Russian editors, protected with $\varepsilon = 0.1$ 
%
\end{itemize}

\end{frame}

%-----------------------------------------------------------------------

\begin{frame}{Other Applications}

\begin{itemize} \setlength\itemsep{2mm}
%
\item Brave uses distributed DP for collecting user analytics
%
\item Spectus published a dashboard with mobility trends during Hurricane Irma
%
\item Microsoft Assistive AI collects user data about Office Tools and suggests automatic replies
%
\item Google uses DP and federated learning for collecting data and training models for improved text selection and copying on Android
%
\end{itemize}

\end{frame}

%-----------------------------------------------------------------------

\section{Standardizations and the Epsilon Registry}

%-----------------------------------------------------------------------

\begin{frame}{Background: Evolution of Information Security}

\textbf{Early days, until the '60s} \vskip2mm

\begin{itemize} \setlength\itemsep{2mm}
%
\item Mostly physical security: focus on access to mainframes 
%
\item Limited threats: computers were rare and isolated
%
\item Computers were mostly used for scientific computations; little or no personal data involved
%
\item Basic enkryption: simple ciphers like Caesar sipher were used to ensure message confidentiality
%
\end{itemize}

\end{frame}

%-----------------------------------------------------------------------

\begin{frame}{Background: Evolution of Information Security}

\textbf{Rise of the networks, '60s - '90s} \vskip2mm

\begin{itemize} \setlength\itemsep{1mm}
%
\item Computer networks and remote access become common
%
\item Unauthorized network / system access and data breaches become popular concerns
%
\item Cybersecurity emerges as an important discipline
%
\item Many remote access protocols with encryption introduced
% 
\item Popular symetric cyphers of 64 bits like DES 
%
\item Public key cryptography comes out
%
\end{itemize}

\end{frame}

%-----------------------------------------------------------------------

\begin{frame}{Background: Evolution of Information Security}

\textbf{The internert era, '00s - present} \vskip2mm

\begin{itemize} \setlength\itemsep{2mm}
%
\item Widespread internet usage, online shopping, personal data sharing
%
\item Surge in cybercrime, viruses, worms, ransomware, phishing, DDOS attacks
%
\item Evolution of security measures: VPNs, firewalls, antiviruses, intrusion detection systems
%
\item Evolution of cryptography: 128 bit cyphers like AES become mandatory
%
\end{itemize}

\end{frame}

%-----------------------------------------------------------------------

\begin{frame}{Information Security Standards}

\begin{itemize} \setlength\itemsep{4mm}
%
\item \textbf{ISO/IEC 27001}: Standards for establishing, implementing, maintaining and improving an ISMS
%
\item \textbf{ISO/IEC 27002}: provides a code of practice for information security controls, offering guidance for implementing security policies
%
\item \textbf{ISO/IEC 27005}: focuses on information security risk management, helping organizations in assessing and managing risks
%
\end{itemize}

\end{frame}

%-----------------------------------------------------------------------

\begin{frame}{Information Privacy vs. Security}

\begin{itemize} \setlength\itemsep{3mm}
%
\item Privacy is inherently a more complex concept: who, what, when, where, why
%
\item Private information is highly context dependent which makes it difficult to protect
%
\item Lack of motivation (unless enforced by law) to protect privacy because of utility (profitability) loss
%
\item Can privacy evolution and standardization follow the same path as security?
%
\end{itemize}

\end{frame}

%-----------------------------------------------------------------------

\begin{frame}{The Epsilon Registry}

\begin{itemize} \setlength\itemsep{3mm}
%
\item DP is being successfully implemented in industry, public sector, academia, etc. 
%
\item Still, little understanding of the optimal $\varepsilon$ values for certain systems, purposes, data types, etc.
%
\item No clear consensus among practicioneers about:
%
	\begin{itemize} \setlength\itemsep{2mm}
	%
	\item How to approach the key implementation decisions
	%
	\item How to decide about the right privacy guarantees
	%
	\item How to choose $\varepsilon$ levels 
	%
	\end{itemize}
%
\end{itemize}

\begin{tikzpicture}[overlay, remember picture]
\node at (current page.north east)[ref] {
\fullcite{dwork_differential_2019} \par};
\end{tikzpicture}

\end{frame}

%-----------------------------------------------------------------------

\begin{frame}{The Epsilon Registry}

Could a public Epsilon Registry help?

\begin{itemize} \setlength\itemsep{3mm}
%
\item Communal body of knowledge about DP implementations
%
\item Used by stakeholders, public servants, academics, etc. 
%
\item Serve as guideline for identifying and adopting judicious DP implementations
%
\end{itemize}

\begin{tikzpicture}[overlay, remember picture]
\node at (current page.north east)[ref] {
\fullcite{dwork_differential_2019} \par};
\end{tikzpicture}

\end{frame}

%-----------------------------------------------------------------------

\begin{frame}{Questions...?}

\pause 

\centering
\textbf{THANK YOU...!}

\end{frame}

%------------------------------------------------------------------------

\end{document}

